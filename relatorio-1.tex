\documentclass[]{article}
\usepackage{amsmath}
\usepackage{authblk}
\usepackage{graphicx}
\graphicspath{ {/home/daniel/Code/puc/} }
\usepackage{multicol}

\usepackage[a4paper, total={6in, 8in}]{geometry}

%opening
\title{Relatório - Medida para ordenar objetos de piores a melhores dado os valores de seus atributos}
\author[1]{Daniel Schlickmann Bastos}
\date{03/03/2021}

\begin{document}
		
	\maketitle
	
	\section{Resumo}
		O objetivo dessa etapa foi desenvolver uma medida de similaridade para o cenário onde é necessário ranquear pessoas de piores a melhores, dado os seus níveis de conhecimento em certos atributos. Pode-se pensar que há uma pessoa com conhecimentos máximos em todos os atributos, e as outras com conhecimentos variáveis, e é preciso atribuir um valor de similaridade para cada dupla, sendo 1 o mais similar e 0, o menos.
		
		Foram ranqueados 64 objetos manualmente (ground truth) e programaticamente, representados por pessoas, cada um com três variáveis (PPT, Word e Excel) e seus respectivos valores categórios ordinais, ditos como conhecimento (Nada, Básico, Intermediário e Avançado). Todos os objetos foram atribuídos um valor entre 0 a 1, onde 0 representa o objeto com menos conhecimento nas três variáveis e o 1, o objeto com mais conhecimento.
		
		No processo de ranqueamento, nota-se uma tendência a valorizar mais o conhecimento especialista em prol do generalista, e que cada nível de conhecimento não é equidistante do outro, ou seja, os pesos dos conhecimentos não são lineares; alguém avançado em Excel é "muito mais" valorizado do que alguém intermediário.
		
		Para conseguir quantificar o conhecimento, a ideia de uma função dilatadora ou compressora foi utilizada para alcançar o valor do peso de cada nível de conhecimento, na prática, a função é uma exponencial, onde o valor do expoente é composto pelo o índice do nível de conhecimento (0, 1, 2 ou 3), e o valor final, o peso.
		
		Por fim, para manter os valores entre 0 e 1, é tirado a média ponderada, onde o denominador é a soma dos pesos de cada conhecimento.
	\section{Proposta}
		A medida para alcançar o valor numérico entre 0 e 1 de cada pessoa dado os seus níveis de conhecimentos de nada, básico, intermediário e avançado em PPT, Word e Excel é dada pela seguinte equação
		
		\begin{center}
			$ S_i = \dfrac{\sum_{k = 0}^N w_{ik}s_{ik}}{\sum_{k = 0}^N w_{ik}} $	
		\end{center}
		
		onde
		
		\begin{itemize}
			
		 \item  $ C = 1.59791167272282585987 $
		
		 \item $ S_k = \{ \mbox{"nada"}, \mbox{"básico"}, \mbox{"intermediário"}, \mbox{"avançado"} \} $
			
	     \item 	$ w_{ik} = C^{r_{ik} - (| S_k | - 1)} $
			
		 \item $ s_{ik} = \dfrac{r_{ik}}{| S_k | - 1} $
		
		 \item $ r_{ik} = \begin{cases}
			        0, & \mbox{se } i_k = S_0 \\ 
			        1, & \mbox{se } i_k = S_1 \\
			        2, & \mbox{se } i_k = S_2 \\
		    		3, & \mbox{se } i_k = S_3 
	        \end{cases} $
        \end{itemize}
    
        
        $ S_k $ representa a sequência dos níveis de conhecimento, $ w_{ik} $, o peso exponencial de cada atributo, $ s_{ij} $, a similaridade do atributo supondo que os conhecimentos têm a mesma valorização e $ r_{ik} $, o índice do nível de conhecimento do atributo na sequência $ S_k $.
        
       A constante $ C $ é igual a aproximadamente 1.5979 somente na escala de 0 a 3, outras escalas têm um valor diferente. Valores de C acima do estimado produzem valores de similaridade onde é possível observar o paradoxo de Simpson, onde a tendência geral é crescente a medida que o nível de cada conhecimento aumenta, porém ao diminuir o nível de conhecimento de uma das variáveis, gera-se um valor de similaridade maior. 
    
        
	\section{Experimento}
		Foram gerados 64 combinações de objetos (pessoas), cada um com um nível diferente de conhecimento em um atributo. Como o nível de importância de cada atributo é o mesmo – um "avançado" em Excel é o mesmo que em Word – o número de combinações diminuí para 20, e estão presentes na tabela abaixo.
		
		\begin{center}
			Tabela 1 - Conhecimentos
			
			\begin{tabular}{|c c c|} 
				\hline
				PPT & Word & Excel \\ [0.5ex] 
				\hline\hline
				nada & nada & nada \\
				\hline
				nada & nada & básico \\
				\hline
				nada & nada & médio \\
				\hline
				nada & nada & avançado \\
				\hline
				nada & básico & básico \\
				\hline
				nada & básico & médio \\
				\hline
				nada & básico & avançado \\
				\hline
				nada & médio & médio \\
				\hline
				nada & médio & avançado \\
				\hline
				nada & avançado & avançado \\
				\hline
				básico & básico & básico \\
				\hline
				básico & básico & médio \\
				\hline
				básico & básico & avançado \\
				\hline
				básico & médio & médio \\
				\hline
				básico & médio & avançado \\
				\hline
				básico & avançado & avançado \\
				\hline
				médio & médio & médio \\
				\hline
				médio & médio & avançado \\
				\hline
				médio & avançado & avançado \\
				\hline
				avançado & avançado & avançado \\
				\hline
			\end{tabular}
		\end{center}
		
		Para cada pessoa (linha) é calculado o valor de similaridade. Um exemplo de como a equação irá se comportar para a pessoa com conhecimentos "nada" em PPT, "intermediário" em Word e "avançado" em Excel segue abaixo.
		
		\begin{center}
		$ S_{ik} = \dfrac{\frac{0}{3} * 1.578^{0 - 3} + \frac{2}{3} * 1.578^{2 - 3} + \frac{3}{3} * 1.578^{3 - 3}}        
		           {1.578^{0 - 3} + 1.578^{2 - 3} + 1.578^{3 - 3}} = 0.75749   {}$
		           
		 		
		\end{center}

 	Para ajudar na compreensão da equação, é possível a visualizar num sistema de coordenadas de três variáveis, entretanto, como há a limitação de três dimensões, é somente possível ter dois atributos (variáveis), o x e y, ao invés de três. O z irá representar o valor final de similaridade. Cada ponto representa cada pessoa (objeto).
 	

	\includegraphics[width=13.2cm, height=6cm]{sim-graph}
	
	
	Observa-se que uma pessoa com nível de conhecimento "básico" em um atributo e "avançado" em outro (0 e 3), tem um valor de z (similaridade) maior do que alguém com conhecimentos "intermediário" e "intermediário" (2 e 2).

	\section{Resultados}
		Ao calcular o valor de similaridade para cada pessoa, é obtido o seguinte resultado:
		
		\begin{center}
			Tabela 2 - Conhecimentos e Similaridade
			
			\begin{tabular}{|c c c | c|} 
				\hline
				PPT & Word & Excel & Similaridade \\ [0.5ex] 
				\hline\hline
				nada & nada & nada & 0.0 \\
				\hline
				nada & nada & básico & 0.1480406615534986 \\
				\hline
				nada & básico & básico & 0.2538892514715101 \\
				\hline
				básico & básico & básico & 0.3333333333333333 \\
				\hline
				nada & nada & médio & 0.3738401507441609 \\
				\hline
				nada & básico & médio & 0.43384788310314315 \\
				\hline
				básico & básico & médio & 0.48137399488683197 \\
				\hline
				nada & médio & médio & 0.5574959455388626 \\
				\hline
				básico & médio & médio & 0.5872225848048434 \\
				\hline
				médio & médio & médio & 0.6666666666666666 \\
				\hline
				nada & nada & avançado & 0.6710516885975281 \\
				\hline
				nada & básico & avançado & 0.6907296861574383 \\
				\hline
				básico & básico & avançado & 0.7071734840774943 \\
				\hline
				nada & médio & avançado & 0.7574959455388627 \\
				\hline
				básico & médio & avançado & 0.7671812164364765 \\
				\hline
				médio & médio & avançado & 0.8147073282201652 \\
				\hline
				nada & avançado & avançado & 0.8908292788721961 \\
				\hline
				básico & avançado & avançado & 0.8908292788721961 \\
				\hline
				médio & avançado & avançado & 0.9205559181381767 \\
				\hline
				avançado & avançado & avançado & 1.0 \\
				\hline
			\end{tabular}
		\end{center}

	\section{Conclusão}
	
	Os resultados obtidos demonstram uma medida que condiz com o ground truth, onde em ambas os conhecimentos especialistas são mais valorizados que os generalistas. A medida, portante, consegue com sucesso atribuir um valor entre 0 e 1 para cada pessoa dependendo de seus conhecimentos.
	
	A única pendência é descobrir como chegar numa fórmula para a base representada por C.
	
	\newpage	
\end{document}

